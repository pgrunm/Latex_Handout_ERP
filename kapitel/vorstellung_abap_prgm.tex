\section{Vorstellung eines ABAP Programms}

\begin{listing}[ht]{}
    \abapcode{code/beispiel.abap}
    \caption{Beispiel eines in ABAP implementierten Taschenrechners}
    \label{listing:abap_taschenrechner}
\end{listing}

Im Listing \ref{listing:abap_taschenrechner} wird ein Taschenrechner in ABAP implementiert. Der Taschenrechner ermöglicht die Nutzung der Grundrechenarten und zeigt ebenso eine grafische Oberfläche zur Auswahl der Operanden an. In Zeile drei und vier werden die zu verarbeiten Zahlen definiert, ab Zeile sechs die Radiobuttons zur Gruppe \textit{gr1}. Beginnend in Zeile 14 wird die Eingabe je nach verwendetem Operator verarbeitet.
\newpage