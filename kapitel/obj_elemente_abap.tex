\section{Objektorientierte Elemente in ABAP}

% Kapitel Klassen
\subsection{Klassen in ABAP}
\begin{itemize}
  \item Zweigeteilte Erstellung einer Klasse.
  \begin{itemize}
    \item Definition enthält z. B. Variablen, Methoden und Sichtbarkeit.
    \item Logik des eigentlichen Programms (Methoden usw.) in der Implementierung.
  \end{itemize}
\end{itemize}

% Beispiel Quellcode Klassen
\begin{listing}[ht]{}
  \abapcode{code/klasse.abap}
  \caption{Vererbung in ABAP}
  \label{listing:abap_klasse}
\end{listing}
Listing \ref{listing:abap_klasse} stellt ein Beispiel zu Klassen in ABAP vor. Die Erstellung ist hier zweigeteilt. Zunächst erfolgt eine Definition z. b. mit Variablen und im Anschluss die Implementierung der Programmlogik\footcite[Vgl.][S. 233]{kellerABAPObjectsIntroduction2002}. 

\subsection{Vererbung in ABAP}
\begin{itemize}
  \item ABAP ermöglicht lediglich eine Superklasse.
  \item Beliebig viele Subklassen. 
  \item Vererbung wird durch das Schlüsselwort 
  \textit{INHERITING} eingeleitet (s. Listing \ref{listing:abap_vererbung}).
\end{itemize}

% Beispiel Quellcode Vererbung
\begin{listing}[ht]{}
  \abapcode{code/vererbung.abap}
  \caption{Vererbung demonstriert an einer Beispielklasse in ABAP}
  \label{listing:abap_vererbung}
\end{listing}
Listing \ref{listing:abap_vererbung} zeigt die Vererbung in ABAP auf. ABAP ermöglicht ermöglicht lediglich eine Superklasse, jedoch beliebig viele Subklassen\footcite[Vgl.][S. 266 ff.]{kellerABAPObjectsIntroduction2002}.

% Kapitel Kapselung 
\subsection{Kapselung in ABAP}
\begin{itemize}
  \item Verwendung sogenannter \textit{Getter} und \text{Setter} Methoden
  \begin{itemize}
    \item Erneut zweigeteilt: Definition und Implementierung.
    \item \textbf{Achtung:} Getter sind nicht unbedingt erforderlich, Read-Only Attribute sind ebenfalls möglich!
  \end{itemize}
  \item Kennzeichnung einer Methode durch Stichwort \textit{Method}.
\end{itemize}

% Beispiel Quellcode Kapselung
% -> Getter & Setter z. B. per Interface (Siehe: https://www.tutorialspoint.com/sap_abap/sap_abap_encapsulation.htm)
%  Oder auch: https://zevolving.com/2011/02/read-only-attribute-vs-getter-methods/
\begin{listing}[ht]{}
  \abapcode{code/kapselung.abap}
  \caption{Beispiele für Getter und Setter Methoden in ABAP}
  \label{listing:abap_getter_setter}
\end{listing}
Dabei sind verschiedene Sichtbarkeiten möglich, darunter Public (öffentlich), Private und Protected\footcite[Vgl.][S. 83]{woodObjectorientedProgrammingABAP2016}.

% Polymorphie in ABAP
\subsection{Polymorphie in ABAP}

% Könnte ein bisschen lang werden...
\begin{listing}[ht]{}
  \abapcode{code/polyphormie.abap}
  \caption{Beispiel von Polymorphie in ABAP}
  \label{listing:abap_polymorphie}
\end{listing}