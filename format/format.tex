\usepackage[ngerman]{babel}			% deutsche Namen/Umlaute
\usepackage[utf8]{inputenc}			% Zeichensatzkodierung

\usepackage{graphicx}				% Einbinden von Bildern
\graphicspath{ {../bilder/} }

\usepackage{color}					% Farben wenn es sein muß
\usepackage[hidelinks]{hyperref}				% Klickbare Verweise und \autoref{label}
\newcommand*{\email}[1]{            % Klickbare eMail-Adressen
    \normalsize\href{mailto:#1}{#1}\par
    }
\usepackage{booktabs}				% "Schöne" Tabellen
\usepackage{amsmath}				% Mathematischer Formelsatz AMS
\usepackage{amsfonts}
\usepackage{csquotes}               % Empfohlener Import für babel/polyglossia
\usepackage{blindtext}              % Einfügen von Blindtext möglich

\usepackage[
    printonlyused                   % nur verwendete Abkürzungen ins Verzeichnis
]{acronym}                          % Abkürzungen
\usepackage[section]{placeins}      % Bilder bleiben im Abschnitt

% Mehr info dazu hier: https://en.wikibooks.org/wiki/LaTeX/Modular_Documents#Subfiles
% \usepackage{subfiles}

% Package minted, see more info here:
% https://github.com/gpoore/minted
% https://www.overleaf.com/learn/latex/Code_Highlighting_with_minted
\usepackage{minted}
\usepackage{caption}

% Literatur
% Siehe hier: https://github.com/James-Yu/LaTeX-Workshop/wiki/Compile#bib-program-and-options
% !BIB program = biblatex
\usepackage[style=ieee, backend=biber]{biblatex}